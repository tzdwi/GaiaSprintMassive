\documentclass[twocolumn]{aastex62}
\pdfoutput=1
\newcommand\aastex{AAS\TeX}
\newcommand\latex{La\TeX}
\newcommand\Gaia{{\it Gaia}}

\usepackage{amsmath,graphicx,multirow}
 
\received{}
\revised{}
\accepted{}
\submitjournal{ApJ}

\shorttitle{The WISE-\Gaia Perspective on Massive Stars}
\shortauthors{Dorn-Wallenstein, Levesque, \& Davenport}


\begin{document}

\title{The WISE-\Gaia Perspective on Massive Stars: Classifying IR Variability Across the Upper HR Diagram}

\correspondingauthor{Trevor Z. Dorn-Wallenstein}
\email{tzdw@uw.edu}

\author[0000-0003-3601-3180]{Trevor Z. Dorn-Wallenstein}
\affiliation{University of Washington Astronomy Department \\
Physics and Astronomy Building, 3910 15th Ave NE  \\
Seattle, WA 98105, USA} 

\author[0000-0003-2184-1581]{Emily M. Levesque}
\affiliation{University of Washington Astronomy Department \\
Physics and Astronomy Building, 3910 15th Ave NE  \\
Seattle, WA 98105, USA}

\author[0000-0002-0637-835X]{James R. A. Davenport}
\altaffiliation{NSF Astronomy and Astrophysics Postdoctoral Fellow; DIRAC Fellow}
\affiliation{University of Washington Astronomy Department \\
Physics and Astronomy Building, 3910 15th Ave NE  \\
Seattle, WA 98105, USA} 
\affiliation{Western Washington University Department of Physics \& Astronomy \\
 516 High St. \\
 Bellingham, WA 98225, USA}

\begin{abstract}

Massive Stars are really cool! Gaia can help us find them. Optical + IR diagnostics are useful for classifying them. WISE gives us awesome variability stuff. We can do some machine learning.


\end{abstract}

\keywords{stars: massive}

\section{Introduction} \label{sec:intro}

This is the introduction. We introduce massive stars, and how their interiors and exteriors evolve. We discuss how that information is encoded into variability. Rotation, mass loss, etc. We discuss the IR. We discuss why WISE+\Gaia$ = <3$.

The paper is laid out as follows: we describe our sample selection using the \Gaia DR2-WISE crossmatch in \S\ref{sec:sample}. We discuss our analysis of the WISE lightcurves in \S\ref{sec:lightcurves}. Our results examining both the coadded \Gaia and WISE data and time-resolved WISE lightcurves are shown in \S\ref{sec:results}. We discuss the implications of our results in \S\ref{sec:discussion} before concluding in \S\ref{sec:conclusion}.

\section{Sample Selection}\label{sec:sample}

\section{Lightcurve Analysis \& Feature Extraction}\label{sec:lightcurves}

\section{Results}\label{sec:results}

\subsection{Photometric Diagnostics}\label{subsec:photometry}

\subsubsection{Distinguishing Between RSGs and AGBs}

\subsection{Variability}\label{subsec:variability}

\subsubsection{Raw Feature Results}

\subsubsection{Machine Learning}

\section{Discussion}\label{sec:discussion}

\section{Summary \& Conclusion}\label{sec:conclusion}



\acknowledgments


This work made use of the following software:

\vspace{5mm}

{\bf Update this}

\software{Astropy v2.0.3 \citep{astropy13,astropy18}, FATS, Matplotlib v2.1.2 \citep{Hunter:2007}, makecite \citep{makecite18}, NumPy v1.14.1 \citep{numpy:2011}, Python 3.5.1}

\bibliography{references}
\bibliographystyle{aasjournal}


\end{document}